%================================================================================
%       Safety Critical Systems Club - Data Safety Initiative Working Group
%================================================================================
%                       DDDD    SSSS  IIIII  W   W   GGGG
%                       D   D  S        I    W   W  G   
%                       D   D   SSS     I    W W W  G  GG
%                       D   D      S    I    WW WW  G   G
%                       DDDD   SSSS   IIIII  W   W   GGG
%================================================================================
%               Data Safety Guidance Document - LaTeX Source File
%================================================================================
%
% Description:
%   Principles and Process section.
%
%================================================================================
\chapter{Principles (Normative)} \label{bkm:principlesprocess}

\dsiwgSectionQuote{Errors using inadequate data are much less than those using no data at all.}{Charles Babbage}

Hawkins \dsiwgTextIT{et.\ al.} established some generic software safety assurance principles, which are commonly referred to as ``4 + 1'' \cite{citation:hawkins2013principles}. Given the close links between software and data it is helpful to consider these principles from a data safety assurance perspective. The results are detailed below, with each principle being considered in turn.

\section{Principle 1: \index{Safety Requirement!Data}\glsfmtplural{data safety requirement} shall be defined to address the data contribution to system hazards}
Data pervades active system operation, as well as the system's specification, realisation, \gls{verification}, \gls{validation}, certification,  training\index{Training!Personnel}, maintenance, and retirement. Moreover, data may be passed from one system to another, sometimes over a significant period of time. Data may be assimilated and converted from prior uses into new uses, or simply used as-is, by many systems. It is stored in media whose storage \gls{integrity} decays. The system context for \index{Safety Requirement!Data}\glspl{data safety requirement} may be specific to a particular system's or process's use of the data, or it may be generalised to a class of related systems. Hence \index{Safety Requirement!Data}\glspl{data safety requirement} are needed for any safety-related system that interacts with data.

\section{Principle 2: the intent of the \index{Safety Requirement!Data}\glsfmtplural{data safety requirement} shall be maintained throughout requirements decomposition}
\index{Safety Requirement!Data}\Glspl{data safety requirement} establish the system's \index{Property!Safety}safety properties for data, for the system's use of data, for the management of data, and for the engineering lifecycle\index{Lifecycle!Engineering} of both the system and its associated data. The system's requirements hierarchy must preserve the intent of the \index{Safety Requirement!Data}\glspl{data safety requirement} (and hence the system's safety-related \index{Property!Data}\glspl{data property}). Moreover, the applied engineering process for both the system's realisation and subsequent lifecycle\index{Lifecycle!System} stages shall demonstrate that the data safety properties are preserved.

\section{Principle 3: \index{Safety Requirement!Data}\glsfmtplural{data safety requirement} shall be satisfied}
Evidence is required that the system satisfies all of the \index{Safety Requirement!Data}\glspl{data safety requirement} imposed on it for all anticipated operating conditions. Moreover, the \index{Safety Requirement!Data}\glspl{data safety requirement} that pertain to the data's lifecycle\index{Lifecycle!Data} outside of the system shall be evidentially demonstrated prior to the system acting on such data, or the system shall be able to adequately defend against unsatisfied \index{Safety Requirement!Data}\glspl{data safety requirement}. In other words, either the data can be shown to demonstrate the required \index{Property!Data}\glspl{data property} prior to being used or the system can implement adequate defences and \glspl{mitigation}\index{Mitigation} against data that does not conform to the required \index{Property!Safety}safety properties.

\section{Principle 4: Hazardous system behaviour arising from the system's use of data shall be identified and mitigated\index{Mitigation}}
This is an intentionally broad statement because data is conceptual and not physical; it is the contextualised use of data that could result in a system hazard. Data safety assurance principle 1 deals with system-level hazards arising from data, whereas Data safety assurance principle 4 is concerned with hazards that arise from the way the system uses its data, that is, whether the system's design and implementation introduce further hazards. An example is a ship navigation system's display of hydrographic chart data, where a wide field display results in small features disappearing (due to image scale) when it is critical that situational awareness of such hazards is maintained.

\section{Principle 4+1: The confidence established in addressing the data safety assurance principles shall be commensurate to the contribution of the data to system risk}
The confidence in the evidence that demonstrates establishment of the first four Data Safety Assurance Principles shall be proportionate to the contribution data (or a particular \index{Artefact, Data}\gls{data artefact}) makes to the system hazards.