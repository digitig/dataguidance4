%================================================================================
%       Safety Critical Systems Club - Data Safety Initiative Working Group
%================================================================================
%                       DDDD    SSSS  IIIII  W   W   GGGG
%                       D   D  S        I    W   W  G   
%                       D   D   SSS     I    W W W  G  GG
%                       D   D      S    I    WW WW  G   G
%                       DDDD   SSSS   IIIII  W   W   GGG
%================================================================================
%               Data Safety Guidance Document - LaTeX Source File
%================================================================================
%
% Description:
%   Principles and Process section.
%
%================================================================================
\cbstart\chapter{Principles (Normative)} \label{bkm:principlesprocess}\cbend

\dsiwgSectionQuote{Errors using inadequate data are much less than those using no data at all.}{Charles Babbage}

Hawkins \dsiwgTextIT{et.\ al.} established some generic software safety assurance principles, which are commonly referred to as ``4 + 1'' \cite{citation:hawkins2013principles}. Given the close links between software and data it is helpful to consider these principles from a data safety assurance perspective. The results are detailed below, with each principle being considered in turn.

\section{Principle 1: \index{Safety Requirement!Data}\glsfmtplural{data safety requirement} shall be defined to address the data contribution to system hazards}
Data pervades active system operation, as well as the system's specification, realisation, \gls{verification}, \gls{validation}, certification,  training\index{Training!Personnel}, maintenance, and retirement. Moreover, data may be passed from one system to another, sometimes over a significant period of time. Data may be assimilated and converted from prior uses into new uses, or simply used as-is, by many systems. It is stored in media whose storage \gls{integrity} decays. The system context for \index{Safety Requirement!Data}\glspl{data safety requirement} may be specific to a particular system's or process's use of the data, or it may be generalised to a class of related systems. Hence \index{Safety Requirement!Data}\glspl{data safety requirement} are needed for any safety-related system that interacts with data. The associated objectives and outputs are shown in \ref{tab:p1_principles}.

\begin{longtable}{|C{\dsiwgColumnWidth{0.06}}|L{\dsiwgColumnWidth{0.47}}|L{\dsiwgColumnWidth{0.47}}|}
	\caption{P1 Objectives and Outputs}\label{tab:p1_principles}
	\\\hline\TableHeadColourC{Id} & \TableHeadColour{Objectives} & \TableHeadColour{Outputs} \\\hline
	\endfirsthead
	\caption{P1 Objectives and Outputs (continued)}
	\\\hline\TableHeadColourC{Id} & \TableHeadColour{Objectives} & \TableHeadColour{Outputs} \\\hline
	\endhead
	\multicolumn{3}{r}{\sl Continued on next page}
	\endfoot\endlastfoot
	\hline
	1.a & System context and intended use SHALL be established. & A description of the system and its intended use. This SHOULD include an estimate of the level of data-related risks.\\
	\hline
	1.b & Key \index{Stakeholder}\glspl{stakeholder} SHALL be identified. & A list of key \index{Stakeholder}\glspl{stakeholder} for data safety activities. \\
	\hline
	1.c & \index{Artefact, Data}\Glspl{data artefact} SHALL be identified. & A collection of \index{Artefact, Data}\glspl{data artefact}, described at an appropriate level of detail.\\
	\hline
	1.d & Missing and concealed data SHALL be considered. & A plan to address potential relevant missing and obscured data.\\
	\hline
	1.e & Interfaces SHALL be defined and managed. & An interface control plan or list of control measures. This MAY include a list of \index{Data!Owner}\glspl{data owner}, linked to \index{Artefact, Data}\glspl{data artefact}.\\
	\hline
	1.f & & A plan for the remaining parts of the \gls{safety assessment}.\\
	\hline\hline
\end{longtable}

\section{Principle 2: the intent of the \index{Safety Requirement!Data}\glsfmtplural{data safety requirement} shall be maintained throughout requirements decomposition}
\index{Safety Requirement!Data}\Glspl{data safety requirement} establish the system's \index{Property!Safety}safety properties for data, for the system's use of data, for the management of data, and for the engineering lifecycle\index{Lifecycle!Engineering} of both the system and its associated data. The system's requirements hierarchy must preserve the intent of the \index{Safety Requirement!Data}\glspl{data safety requirement} (and hence the system's safety-related \index{Property!Data}\glspl{data property}). Moreover, the applied engineering process for both the system's realisation and subsequent lifecycle\index{Lifecycle!System} stages shall demonstrate that the data safety properties are preserved. The associated objectives and outputs are shown in \ref{tab:p2_principles}.

\begin{longtable}{|C{\dsiwgColumnWidth{0.06}}|L{\dsiwgColumnWidth{0.47}}|L{\dsiwgColumnWidth{0.47}}|}
	\caption{P2 Objectives and Outputs}\label{tab:p2_principles}
	\\\hline\TableHeadColourC{Id} & \TableHeadColour{Objectives} & \TableHeadColour{Outputs} \\\hline
	\endfirsthead
	\caption{P2 Objectives and Outputs (continued)}
	\\\hline\TableHeadColourC{Id} & \TableHeadColour{Objectives} & \TableHeadColour{Outputs} \\\hline
	\endhead
	\multicolumn{3}{r}{\sl Continued on next page}
	\endfoot\endlastfoot
	\hline
	2.a & Historical data-related accidents and incidents SHALL be reviewed. &\\
	\hline
	2.b & Unintended behaviour resulting from data SHALL be identified and analysed\cbstart, specifically considering black swan data, dragon king data, perfect storm data and Pudding Lane data.& A missing and obscured data resilience report.\cbend\\
	\hline
	2.c & Risks SHALL be identified and linked to \index{Artefact, Data}\glspl{data artefact} and \index{Property!Data}\glspl{data property}. & A list of risks, linked to \index{Artefact, Data}\glspl{data artefact} and \index{Property!Data}\glspl{data property}.\\
	\hline
	2.d & & A description of the process used for risk identification.\\
	\hline
	2.e & & An updated plan for the remaining parts of the \index{Safety Assessment!Data}\gls{safety assessment}, if required.\\
	\hline
	2.f & & An unintended behaviour analysis report.\\
	\hline
\end{longtable}
\section{Principle 3: \index{Safety Requirement!Data}\glsfmtplural{data safety requirement} shall be satisfied}
Evidence is required that the system satisfies all of the \index{Safety Requirement!Data}\glspl{data safety requirement} imposed on it for all anticipated operating conditions. Moreover, the \index{Safety Requirement!Data}\glspl{data safety requirement} that pertain to the data's lifecycle\index{Lifecycle!Data} outside of the system shall be evidentially demonstrated prior to the system acting on such data, or the system shall be able to adequately defend against unsatisfied \index{Safety Requirement!Data}\glspl{data safety requirement}. In other words, either the data can be shown to demonstrate the required \index{Property!Data}\glspl{data property} prior to being used or the system can implement adequate defences and \glspl{mitigation}\index{Mitigation} against data that does not conform to the required \index{Property!Safety}safety properties. The associated objectives and outputs are shown in \ref{tab:p3_principles}.

\begin{longtable}{|C{\dsiwgColumnWidth{0.06}}|L{\dsiwgColumnWidth{0.47}}|L{\dsiwgColumnWidth{0.47}}|}
	\caption{P3 Objectives and Outputs}\label{tab:p3_principles}
	\\\hline\TableHeadColourC{Id} & \TableHeadColour{Objectives} & \TableHeadColour{Outputs} \\\hline
	\endfirsthead
	\caption{P3 Objectives and Outputs (continued)}
	\\\hline\TableHeadColourC{Id} & \TableHeadColour{Objectives} & \TableHeadColour{Outputs} \\\hline
	\endhead
	\multicolumn{3}{r}{\sl Continued on next page}
	\endfoot\endlastfoot
	\hline
	3.a & \Glspl{dsal} SHALL be established. & A \gls{dsal} for each risk identified in the previous phase.\\
	\hline
	3.b & \Glspl{dsal} SHALL be justified. & A justification for each DSAL identified\\
	\hline
	3.c & \Glspl{dsal} SHALL be incorporated into system safety activities.& \\
	\hline
	3.d & & An updated plan for the remaining parts of the \index{Safety Assessment!Data}\gls{safety assessment}, if required.\\
	\hline
\end{longtable}

\section{Principle 4: Hazardous system behaviour arising from the system's use of data shall be identified and mitigated\index{Mitigation}}
This is an intentionally broad statement because data is conceptual and not physical; it is the contextualised use of data that could result in a system hazard. Data safety assurance principle 1 deals with system-level hazards arising from data, whereas Data safety assurance principle 4 is concerned with hazards that arise from the way the system uses its data, that is, whether the system's design and implementation introduce further hazards. An example is a ship navigation system's display of hydrographic chart data, where a wide field display results in small features disappearing (due to image scale) when it is critical that situational awareness of such hazards is maintained. The associated objectives and outputs are shown in \ref{tab:p4_principles}.

\begin{longtable}{|C{\dsiwgColumnWidth{0.06}}|L{\dsiwgColumnWidth{0.47}}|L{\dsiwgColumnWidth{0.47}}|}
	\caption{P4 Objectives and Outputs}\label{tab:p4_principles}
	\\\hline\TableHeadColourC{Id} & \TableHeadColour{Objectives} & \TableHeadColour{Outputs} \\\hline
	\endfirsthead
	\caption{P4 Objectives and Outputs (continued)}
	\\\hline\TableHeadColourC{Id} & \TableHeadColour{Objectives} & \TableHeadColour{Outputs} \\\hline
	\endhead
	\multicolumn{3}{r}{\sl Continued on next page}
	\endfoot\endlastfoot
	\hline
	4.1 & & A record of the agreed \index{Response}\glspl{response} to each of the identified risks, along with supporting justification\index{Justification, Safety}.\\
	\hline
	4.2 & \index{Safety Requirement!Data}\Glspl{data safety requirement} SHALL be established and elaborated. & A list of \index{Safety Requirement!Data}\glspl{data safety requirement} that follow from these \index{Response}\glspl{response}.\\
	\hline
	4.3 & Methods used to provide data safety assurance SHALL be defined and implemented. & A record of the \index{Treatment!Risk}\gls{treatment} adopted for each of the identified risks, including evidence that the \index{Treatment!Risk}\gls{treatment} has been successfully implemented.\\
	\hline
	4.4 & Compliance with \index{Safety Requirement!Data}\glspl{data safety requirement} SHALL be demonstrated.	& An assessment as to whether the risk has been suitably mitigated\index{Mitigation} (and, if not, plans for further \gls{mitigation}\index{Mitigation} activities).\\
	\hline
\end{longtable}

\section{Principle 5: The confidence established in addressing the data safety assurance principles shall be commensurate to the contribution of the data to system risk}
(Called Principle 4+1 by Hawkins.) The confidence in the evidence that demonstrates establishment of the first four Data Safety Assurance Principles shall be proportionate to the contribution data (or a particular \index{Artefact, Data}\gls{data artefact}) makes to the system hazards. The associated objectives and outputs are shown in \ref{tab:p5_principles}.

\begin{longtable}{|C{\dsiwgColumnWidth{0.06}}|L{\dsiwgColumnWidth{0.47}}|L{\dsiwgColumnWidth{0.47}}|}
	\caption{P1 Objectives and Outputs}\label{tab:p5_principles}
	\\\hline\TableHeadColourC{Id} & \TableHeadColour{Objectives} & \TableHeadColour{Outputs} \\\hline
	\endfirsthead
	\caption{P1 Objectives and Outputs (continued)}
	\\\hline\TableHeadColourC{Id} & \TableHeadColour{Objectives} & \TableHeadColour{Outputs} \\\hline
	\endhead
	\multicolumn{3}{r}{\sl Continued on next page}
	\endfoot\endlastfoot
	\hline
	5.1 & TBD & TBD\\
	\hline
\end{longtable}
\todo{expand dark data, make into objective}
\todo{Add definitions for cygnology types}
\todo{Identify process for dark data}
