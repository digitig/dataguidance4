%================================================================================
%       Safety Critical Systems Club - Data Safety Initiative Working Group
%================================================================================
%                       DDDD    SSSS  IIIII  W   W   GGGG
%                       D   D  S        I    W   W  G   
%                       D   D   SSS     I    W W W  G  GG
%                       D   D      S    I    WW WW  G   G
%                       DDDD   SSSS   IIIII  W   W   GGG
%================================================================================
%               Data Safety Guidance Document - LaTeX Source File
%================================================================================
%
% Description:
%   This file contains the introductory text to the main guidance document.
%
%================================================================================

%
%This is the first page of the main document, we will use Arabic numerals from
%here on in. We also want a standard page style for these chapters until we get
%to the appendices, so make it so.
%
\setcounter{page}{1}
\pagestyle{Standard} % Page style for the rest of the document
\thispagestyle{FirstPage}

\chapter{Introduction (Informative)} \label{bkm:introduction}

\dsiwgSectionQuote{We're entering a new world in which data may be more important than software.}{Tim O'Reilly}

\section{Aim and Scope}
This guidance document aims to:
\begin{itemize}
	\item describe the data safety problem;
	\item provide methods for identifying and analysing levels of risk; and
	\item recommend methods and approaches for evaluating and treating those risks.
\end{itemize}

It has been written for a wide readership. Its target audience is all those who have an interest in or a responsibility for safety-related data within systems, including managers, developers, safety engineers, assurers (including independent safety auditors), regulators, and operators. 

The document is also intended to cover a number of different sectors. It identifies a wide spectrum of safety-related data that exists in many forms within systems, from specification and requirements data to maintenance and disposal data, and everything in between. In particular, this document is not just concerned with numerical or well-structured data used during system operation.

While they are considered mature enough to be useful, the contents of the document represent current thoughts on what is a complex and evolving area. Furthermore, to allow it to be produced within a reasonable timescale, this edition focuses on key items. It is not intended to be exhaustive. For example, this guidance document does not consider issues relating to staff competence or organizational structure.

\section{Intended Relationship to Other Documents}
This document is intended to be used as a supplement to existing standards and norms that are relevant to the scope of the work being undertaken. It may be used to provide a deeper insight into the risks that data poses to the project team's outputs, allowing them to produce credible improvements to the safety argument. Where a standard or norm sets out specific data-related objectives then, unless agreed otherwise with the regulator or safety duty holder, they shall take precedence over the guidance provided herein.

In the longer term, the hope is that future standards and norms will take up relevant concepts, approaches and methods from those in this document. The \gls{dsiwg} also hopes that organizations will include the concepts, approaches and methods in their own safety management processes.

\clearpage %Manual page break
\section{Normative, Informative and Discursive Text}
Three types of text are used within this guidance document:
\begin{description}
	\item[Normative] text, which is prescriptive. Typically, this text is restricted to describing objectives and outputs.
	\item[Informative] text, which is descriptive text that is closely linked to the normative text. Typically, this text provides a suggested way by which compliance with the normative text may be achieved, but alternative means of compliance are possible.
	\item[Discursive] text, which contains discussions that are relevant to the general topic of data safety, but which are not closely linked to the normative text. A discussion on the relationship between data and software is an example of such text. Descriptions of historical incidents and accidents are another.
\end{description}

Each section and appendix of this guidance document contains a single text type. The relevant type is indicated in the section or appendix title.

\section{Compliance}
There may be occasions when it is desirable or necessary to make a claim of compliance against the objectives listed in this document. Such a claim may be required, for example, if this document is explicitly included as a normative reference from a formal standard. Alternatively, it may be required as part of an organization's internal processes.

To facilitate compliance claims, the following terminology is used within the normative parts of this guidance document:
\begin{description}
	\item[SHALL] denotes items where evidence of compliance must be provided in order to claim compliance with this guidance document.
	\item[SHOULD] denotes items where, in some circumstances, there may be valid reasons for not complying with a particular item. The full implications of non-compliance must be understood, documented and approved in order to claim compliance with this guidance document.
	\item[MAY] denotes items that are optional. These may be advantageous in some circumstances but not in others. Organizations are free to adopt any approach to these items without the need for further justification.
\end{description}

The terms have their normal English meanings in discursive and descriptive sections.
