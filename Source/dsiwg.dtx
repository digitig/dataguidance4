% \iffalse meta-comment
%
% Copyright (C) 2025 Safety Critical Systems Club
%
% This work is licensed under the Creative Commons Attribution 4.0 International License. To view a copy
% of this license, visit http://creativecommons.org/licenses/by/4.0/ or send a letter to Creative Commons,
% PO Box 1866, Mountain View, CA 94042, USA. You are free to share the material in any form and adapt the
% material for any purpose providing you attribute the material to the Safety-Critical Systems Club (SCSC)
% Data Safety Initiative Working Group, reference the source material, include the licence details above, and
% indicate if any changes were made. See the license for full details.
%
% \fi
% \iffalse
%<package>\NeedsTeXFormat{LaTeX2e}[2023-11-01]
%<package>\ProvidesPackage{dsiwg}
%<package> [2025-04-02 v0.1 dsiwg style file]
%
%<*driver>
\documentclass{ltxdoc}
\usepackage{dsiwg}
\EnableCrossrefs
\CodelineIndex
\RecordChanges
\begin{document}
	\DocInput{dsiwg.dtx}
\end{document}
%</driver>
% \fi
% \changes{v0.1}{2025-04-02}{Initial version}
% \GetFileInfo{dsiwg.sty}
% \DoNotIndex{\#,\$,\%,\&,\@,\\,\{,\},\^,\_,\~,\ }
% \DoNotIndex{\@ne}
% \DoNotIndex{\advance,\begingroup,\catcode,\closein}
% \DoNotIndex{\closeout,\day,\def,\edef,\else,\empty,\endgroup}
% \title{The \textsf{dsiwg} package\thanks{This document
% corresponds to \textsf{dsiwg}~\fileversion,
% dated~\filedate.}}
% \author{Tim Rowe \\ \texttt{digitig@gmail.com}}
%
% \maketitle
% \section{Introduction}
% The \texttt{dsiwg} package defines a standard appearance for guidance documents produced by the Data
% Safety Initiative Working Group, part of the Safety-Critical Systems Club. It was introduced at version 
% 4.0 of the Data Safety Guidance, but draws extensively on \texttt{Template.tex} used for earlier editions.
%
% The motivation for moving from an included \LaTeXe source file to a style file is that at version 4.0 
% the guidance was split into three separate volumes. Using a style simplifies document organisation. (Using a class
% might make it even simpler, but that would be more work to produce.)
% \section{Usage}
% Invoke the dsiwg style by including |\usepackage| \oarg{options} |dsiwg| to the document preamble, where
% \textit{options} is a comma-separated list of any of the following:
% \begin{description}
% \item[changebars] If present, show changebars marked by |\cbstart| and |\cbend|;
% \item[covers] If present, include front and back covers;
% \item[draft] If present, watermark the document with the text ``DRAFT'';
% \item[hyperref] If present, produce coloured hyperref links; and
% \item[simple] If present, produce a simple text-only version of the document. Overrules all other options.
% \end{description}
% Each option makes a corresponding \LaTeXe switch available in the document. The switch is of the form
% |if_dsiwg_|\oarg{options}, for example |if_dsiwg_draft|.
% \MaybeStop{\PrintIndex}
% \section{Implementation}
% \subsection{Page Definitions}
% Include |fancyheadr| to allow page definitions.
%~~~~\begin{macrocode}
\RequirePackage{fancyhdr}
%~~~~\end{macrocode}

% Define page styles.
%~~~~\begin{macrocode}
\fancypagestyle{FirstPageFrontCover}{\fancyhf{}
  \fancyhead[L]{}
  \fancyfoot[L]{}
  \renewcommand\headrulewidth{0pt}
  \renewcommand\footrulewidth{0pt}
  \renewcommand\thepage{}
  \vspace*{6cm}
  \pagenumbering{alph}
  \renewcommand\thepage{\alph{page}}
}
%~~~~\end{macrocode}
% \Finale
\endinput
