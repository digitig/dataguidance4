%================================================================================
%       Safety Critical Systems Club - Data Safety Initiative Working Group
%================================================================================
%                       DDDD    SSSS  IIIII  W   W   GGGG
%                       D   D  S        I    W   W  G   
%                       D   D   SSS     I    W W W  G  GG
%                       D   D      S    I    WW WW  G   G
%                       DDDD   SSSS   IIIII  W   W   GGG
%================================================================================
%               Data Safety Guidance Document - LaTeX Source File
%================================================================================
%
% Description:
%   Acronyms, Definitions and Glossary section.
%
%================================================================================
\chapter{Acronyms, Definitions and Glossary (Discursive)} \label{bkm:acronyms}

\dsiwgSectionQuote{The plural of anecdote is not data.}{Mark Berkoff}

%
%Uncomment the following to have LaTeX automatically list the acronyms and
%glossary entries used in this document, as indicated by the \gls, \glspl. (All defined in the file 'glossary.tex'
% Note: some of these now done, where the Perfectit tool identified issues
% in the existing document. I think it should be \acrlong, not \acrfull.
% New command \acrentry{} introduced to reduce repetition
%
%THIS WILL BE LEFT UNTIL THE NEXT VERSION (2.X OR 3.0)
%
% Too many hyperlinks if we reference every mention of "data".
\glsadd{data}

% Duplicate normative entries in discursive list
\glsadd{data artefact - disc}
\glsadd{data owner - disc}
\glsadd{safety assessment - disc}
\glsadd{data property - disc}
\glsadd{data safety assurance level - disc}
\glsadd{data safety requirement - disc}
\glsadd{response - disc}
\glsadd{stakeholder - disc}
\glsadd{treatment - disc}

%\printglossary[type=acronym]
%\printglossary
\section{Acronyms}
\printglossary[type=acronym]
%\printglossary
\section{Definitions and Glossary}
\printglossary[style=altlist]

The normative list of definitions is at \autoref{bkm:definitionsabbreviations}. Normative definitions have been repeated here for convenience.
