%================================================================================
%       Safety Critical Systems Club - Data Safety Initiative Working Group
%================================================================================
%                       DDDD    SSSS  IIIII  W   W   GGGG
%                       D   D  S        I    W   W  G   
%                       D   D   SSS     I    W W W  G  GG
%                       D   D      S    I    WW WW  G   G
%                       DDDD   SSSS   IIIII  W   W   GGG
%================================================================================
%               Data Safety Guidance Document - LaTeX Source File
%================================================================================
%
% Description:
%   Glossary file defining all of the terms/acrnoyms used within the guidance
%   document.
%
%================================================================================

%
%This is very much a work-in-progress and not all entries are present. But, like
%the references using BibTeX, if used properly LaTeX will manage all of our
%acronyms and glossary entries for us, if we choose to go down that route.
%This is merely highlighting what is possible for now.
%

%A
\newacronym{aal}{AAL}{above aerodrome level}
\newacronym{adiru}{ADIRU}{air data inertial reference unit}
\newacronym{ai}{AI}{artificial intelligence}
\newacronym{aoa}{AoA}{angle of attack}
\newacronym{arp}{ARP}{aerospace recommended practice}
\newacronym{arq}{ARQ}{automatic repeat-request}
\newacronym{atm}{ATM}{air traffic management}
\newacronym{atsb}{ATSB}{Australian Transport Safety Bureau}
%B
\newacronym{bit}{BIT}{built in test}
\newacronym{bite}{BITE}{built in test equipment}
%C
\newacronym{ccrc}{CCRC}{Criminal Cases Review Commission}
\newacronym{cctv}{CCTV}{closed circuit television}
\newacronym{cfit}{CFIT}{controlled flight into terrain}
\newacronym{cns}{CNS}{communications, navigation, and surveillance}
\newacronym{cod}{CoD}{certificate of design}
\newacronym{cots}{COTS}{commercial off-the-shelf}
\newacronym{cpu}{CPU}{central processing unit}
\newacronym{crc}{CRC}{cyclic redundancy check}
\newacronym{csv}{CSV}{comma separated variable}
\newacronym{ct}{CT}{computed tomography}
%D
\newacronym{dcb}{DCB}{data coordination board}
\newacronym{dme}{DME}{distance measuring equipment}
\newacronym{doc}{DoC}{Department of Corrections}
\newacronym{dracas}{DRACAS}{defect reporting and corrective action system}
\newacronym{dsal}{DSAL}{\gls{data safety assurance level}}
\newacronym{dsiwg}{DSIWG}{Data Safety Initiative Working Group}
\newacronym{dsg}{DSG}{data safety guidance}
\newacronym{dsmp}{DSMP}{data safety management plan}
%E
\newacronym{ecs}{ECS}{electronic chart system}
\newacronym{ecu}{ECU}{electronic control unit}
\newacronym{edm}{EDM}{entry demonstrator module}
\newacronym{egpws}{EGPWS}{enhanced ground proximity warning system}
\newacronym{ehr}{EHR}{electronic health record}
\newacronym{enc}{ENC}{electronic navigational chart}
\newacronym{esa}{ESA}{European Space Agency}
%F
\newacronym{faa}{FAA}{Federal Aviation Administration}
\newacronym{fcpc}{FCPC}{flight control primary computer}
\newacronym{fdal}{FDAL}{functional design assurance level}
\newacronym{fhi}{FHI}{Future of Humanity Institute}
\newacronym{fmgs}{FMGS}{flight management guidance system}
\newacronym{fms}{FMS}{flight management system}
%G
\newacronym{gcs}{GCS}{ground control station}
\newacronym{gp}{GP}{general practitioner}
\newacronym{gps}{GPS}{Global Positioning System}
\newacronym{gtols}{GTOLS}{\glsfmtshort{gps} take-off and landing system}
%H
\newacronym{hse}{HSE}{Health and Safety Executive}
\newacronym{hazop}{HAZOP}{hazard and operability study}
\newacronym{hums}{HUMS}{health and usage monitoring system}
%I
\newacronym{ibm}{IBM}{International Business Machines Corporation}
\newacronym{icao}{ICAO}{International Civil Aviation Organization}
\newacronym{icd}{ICD}{interface control document}
\newacronym{idal}{IDAL}{item development assurance level}
\newacronym{imc}{IMC}{instrument meteorological conditions}
\newacronym{imf}{IMF}{International Monetary Fund}
\newacronym{imu}{IMU}{inertial measurement unit}
\newacronym{ip}{IP}{internet protocol}
\newacronym{irs}{IRS}{inertial reference system}
\newacronym{iso}{ISO}{International Standards Organization}
\newacronym{iv}{IV}{intravenous}
%J
%K
%L
\newacronym{lhr}{LHR}{London Heathrow airport}
\newacronym{lfr}{LFR}{live facial recognition}
\newacronym{llm}{LLM}{large language model}
%M
\newacronym{maib}{MAIB}{Maritime Accident Investigation Board (\textit{or} Branch)}
\newacronym{mca}{MCA}{Maritime and Coastguard Agency}
\newacronym{mcas}{MCAS}{manoevring characteristics augmentation system}
\newacronym{ml}{ML}{machine learning}
\newacronym{msaw}{MSAW}{minimum safe altitude warning}
%N
\newacronym{nan}{NaN}{not a number}
\newacronym{notam}{NOTAM}{notice to airmen}
%O
\newacronym{odr}{ODR}{organizational data risk}
\newacronym{oed}{OED}{Oxford English Dictionary}
\newacronym{oow}{OOW}{officer of the watch}
\newacronym{osi}{OSI}{open system interconnection}
%P
\newacronym{pai}{PAI}{Partnership on \glsxtrshort{ai}}
\newacronym{pcr}{PCR}{polymerase chain reaction}
\newacronym{pocl}{POCL}{Post Office Counters Limited}
%Q
%R
\newacronym{rda}{RDA}{radar Doppler altimeter}
%S
\newacronym{saiwg}{SAIWG}{Safe \gls{ai} Working Group}
\newacronym{sar}{SAR}{search and rescue}
\newacronym{saswg}{SASWG}{Safety of Autonomous Systems Working Group}
\newacronym{scsc}{SCSC}{Safety-Critical Systems Club}
\newacronym{sil}{SIL}{safety \gls{integrity} level}
\newacronym{smp}{SMP}{safety management plan}
\newacronym{sop}{SOP}{standard operating procedure}
\newacronym{sotif}{SOTIF}{safety of the intended functionality}
\newacronym{spm}{SPM}{subpostmaster}
\newacronym{sss}{SSS}{Safety-critical Systems Symposium}
%T
\newacronym{tcas}{TCAS}{traffic collision avoidance system}
%U
\newacronym{uas}{UAS}{unmanned air system}
\newacronym{usb}{USB}{universal serial bus}
%V
\newacronym{vhf}{VHF}{Very High Frequency}
\newacronym{vor}{VOR}{\glsxtrshort{vhf} omnidirectional range}
\newacronym{vmc}{VMC}{visual meteorological conditions}
\newacronym{vms}{VMS}{voyage management system}
%W
%X
\newacronym{xml}{XML}{extensible markup language}
%Y
%Z

%%%%%%%%%%%%%%%%%%%%%%%%%%%%%%%%%%%%%%%%%%
%
% Normative glossary entries
%
%%%%%%%%%%%%%%%%%%%%%%%%%%%%%%%%%%%%%%%%%%

\newglossaryentry{data artefact}{
	type={normative},
	name={artefact, data},
	text={data artefact},
	description={An item, or collection of items, that provides a useful perspective on data generated, processed or consumed by a system.}}

\newglossaryentry{data owner}{
	type={normative},
	name={owner, data},
	text={data owner},
	description={The individual or organization responsible for a particular \index{Artefact, Data}\gls{data artefact} or collection of \glspl{data artefact}.}}

\newglossaryentry{safety assessment}{
	type={normative},
	name={safety assessment, data},
	text={data safety assessment}, 
	description={ The process of explicitly considering data as part of a system safety assessment, via the means of \glspl{data artefact}, Data Properties and \glspl{dsal}.}}

\newglossaryentry{data property}{
	type={normative},
	name={property, data},
	text={data property},
	plural={data properties},
	description={A characteristic that can be exhibited by a \index{Artefact, Data}\gls{data artefact}.}}

\newglossaryentry{data safety assurance level}{
	type={normative},
	name={safety assurance level, data},
	text={data safety assurance level},
	description={An indication of the level of rigour with which relevant \glspl{data property} should be demonstrated for appropriate \index{Artefact, Data}\glspl{data artefact}.}}

\newglossaryentry{data safety requirement}{
	type={normative},
	name={safety requirement, data},
	text={data safety requirement},
	description={A requirement to implement an approach specifically designed to achieve, maintain or demonstrate a \gls{data property} (or \glspl{data property}) for a given \index{Artefact, Data}\glspl{data artefact} (or Artefacts).}}

\newglossaryentry{response}{
	type={normative},
	name={response},
	description={The way in which an identified risk is addressed; possible responses include avoid / eliminate, treat, or accept as sufficiently low.}}

\newglossaryentry{stakeholder}{
	type={normative},
	name={stakeholder},
	description={An individual or organization that has some relationship to the system, possibly including a power of veto.}}

\newglossaryentry{treatment}{
	type={normative},
	name={treatment},
	description={An action taken to reduce or control risk. This might be \gls{mitigation} of the risk or elimination of the hazard.}}

%%%%%%%%%%%%%%%%%%%%%%%%%%%%%%%%%%%%%%%%%%
%
% Duplicate normative entries for inclusion in discursive list.
% (Would like to automate this)
%
%%%%%%%%%%%%%%%%%%%%%%%%%%%%%%%%%%%%%%%%%%
\newglossaryentry{data artefact - disc}{
	name={artefact, data},
	text={data artefact},
	description={(Normative) An item, or collection of items, that provides a useful perspective on data generated, processed or consumed by a system.}}

\newglossaryentry{data owner - disc}{
	name={owner, data},
	text={data owner},
	description={(Normative) The individual or organization responsible for a particular \index{Artefact, Data}\gls{data artefact} or collection of \glspl{data artefact}.}}

\newglossaryentry{safety assessment - disc}{
	name={safety assessment, data},
	text={data safety assessment}, 
	description={(Normative) The process of explicitly considering data as part of a system safety assessment, via the means of \glspl{data artefact}, Data Properties and \glspl{dsal}.}}

\newglossaryentry{data property - disc}{
	name={property, data},
	text={data property},
	plural={data properties},
	description={(Normative) A characteristic that can be exhibited by a \index{Artefact, Data}\gls{data artefact}.}}

\newglossaryentry{data safety assurance level - disc}{
	name={safety assurance level, data},
	text={data safety assurance level},
	description={(Normative) An indication of the level of rigour with which relevant \glspl{data property} should be demonstrated for appropriate \index{Artefact, Data}\glspl{data artefact}.}}

\newglossaryentry{data safety requirement - disc}{
	name={safety requirement, data},
	text={data safety requrement},
	description={(Normative) A requirement to implement an approach specifically designed to achieve, maintain or demonstrate a \gls{data property} (or \glspl{data property}) for a given \index{Artefact, Data}\glspl{data artefact} (or Artefacts).}}

\newglossaryentry{response - disc}{
	name={response},
	description={(Normative) The way in which an identified risk is addressed; possible responses include avoid / eliminate, treat, or accept as sufficiently low.}}

\newglossaryentry{stakeholder - disc}{
	name={stakeholder},
	description={(Normative) An individual or organization that has some relationship to the system, possibly including a power of veto.}}

\newglossaryentry{treatment - disc}{
	name={treatment},
	description={(Normative) An action taken to reduce or control risk.}}



%%%%%%%%%%%%%%%%%%%%%%%%%%%%%%%%%%%%%%%%%%
%
% Discursive glossary entries
%
%%%%%%%%%%%%%%%%%%%%%%%%%%%%%%%%%%%%%%%%%%

\longnewglossaryentry{accuracy}{%
	name={accuracy},%
	plural={accuracies}}%
	{\begin{itemize}
	\item Closeness of agreement between a test result and the accepted reference value. Note that a test result can be observations or measurements. ISO\ 19113:2005 \cite{citation:ISO19113} 
	\item A degree of conformance between the estimated or measured value and the true value. (EU) No 73/2010 \cite{citation:EU732010}
	\item (Temporal)  Correctness of the temporal references of an item (reporting of error in time measurement). Correctness of ordered events or sequences, if reported. \Gls{validity} of data with respect to time. ISO\ 19138:2006 \cite{citation:ISO19138}	
	\end{itemize}
}
% Data Artefact is also in the normative definitions, and should not be repeated.
%\newglossaryentry{artefact}{name={(data) artefact},description={ An item, or collection of items, that provides a useful perspective on data
%		generated, processed or consumed by a system.}}
% I've split data assurance level and software assurance level

\newglossaryentry{data assurance level}{
	name={data assurance level},
	description={The required \index{Assurance Level}assurance level for the aeronautical data process is identified, based on the overall system architecture through allocation of risk determined using a preliminary \index{Safety Assessment!System}system safety assessment. RTCA/DO-200A \cite{citation:ED76}}}

\newglossaryentry{accessibility}{name={accessibility},plural={accessibilities},description={Property that the data is visible only to those that should see it.}}

\longnewglossaryentry{adaptation data}{%
	name={adaptation data},%
	plural={adaptation data}}%
	{Data used to customise elements of the system for their designated purpose.
		Adaptation data\index{Adaptation Data} is used to customise elements of the system for its designated purpose at a specific location.
		These systems are often configured to accommodate site-specific characteristics.
		These site dependencies are developed into sets of adaptation data\index{Adaptation Data}.
		Adaptation data\index{Adaptation Data} includes data that configures the software for a given geographical site, and data that configures a workstation to the preferences and / or functions of an operator.
		Examples include, but are not limited to:
		\begin{description}
			\item[Geographical Data:] latitude and longitude of a radar site.
			\item[Environmental Data:] operator selectable data to provide their specific preferences.
			\item[Airspace Data:] sector-specific data.
			\item[Procedures:] operational customisation to provide the desired operational role.
		\end{description}
		Adaptation data\index{Adaptation Data} may take the form of changes to either database parameters or take the form of pre-programmed options.
		In some cases, adaptation data\index{Adaptation Data} involves re-linking software code to include different libraries.
		Note that this should not be confused with recompilation in which a completely new version of the code is generated. Based on ED-153 \cite{citation:ED153}}

\longnewglossaryentry{aeronautical data}{
	name={aeronautical data}, 
	plural={aeronautical data}}%
	{\begin{itemize}
	\item A representation of aeronautical facts, concepts or instructions in a formalized manner suitable for communication, interpretation or processing. (EU) No 73/2010 \cite{citation:EU732010}
	\item Data used for aeronautical applications such as navigation, flight planning, flight simulators, terrain awareness, and other purposes. RTCA/DO-178C \cite{citation:ED12C}
	\end{itemize}}
\newglossaryentry{analysability}{name={analysability},description={The data (including any \gls{metadata}) is of a suitable size, type and format to enable it be usefully analysed}}

\newglossaryentry{availability}{
	name={availability},
	plural={availabilities},
	description={The property of being accessible and usable upon demand by an authorized entity. ISO\ 27001:2013 \cite{citation:ISO27001:2013}}}

\longnewglossaryentry{completeness}{
	name={completeness}}
	{\begin{itemize}
			\item Property of having every necessary part or element.
			\item Completeness of the data provided. RTCA/DO-200A
	\end{itemize}}
\newglossaryentry{software assurance level}{
	name={assurance level, software},
	text={software assurance level},
	description={An indication of how much assurance is required (commensurate to risk) before deploying software into an operational system. J Spriggs, based on (EC) No 482/2008 \cite{citation:EC4822008}}}

\longnewglossaryentry{configuration data}{
	name={configuration data}, 
	plural={configuration data}}%
	{\begin{itemize}
		\item Data that configures a generic software system to a particular instance of its use. (EC) No 482/2008 \cite{citation:EC4822008}
		\item Data that configures a generic software system to a particular instance of its use (e.g., data for flight data processing system for a particular airspace, by setting the positions of airways, reporting points, navigation aids, airports and other elements important to air navigation). ED-153 \cite{citation:ED153}Data that configures a generic software system to a particular instance of its use (e.g., data for flight data processing system for a particular airspace, by setting the positions of airways, reporting points, navigation aids, airports and other elements important to air navigation). ED-153 \cite{citation:ED153}
	\end{itemize}}

\newglossaryentry{confidentiality}{
	name={confidentiality},
	description={The property that \gls{information} is not made available or disclosed to unauthorized individuals, entities, or processes. ISO27001:2013 \cite{citation:ISO27001:2013}}}

\newglossaryentry{consistency}{
	name={consistency, data},
	text={consistency},
	description={The property that the data adheres to a common world view (e.g., units).} }

\newglossaryentry{continuity}{
	name={continuity, data},
	text={continuity},
	description={The property that the data is continuous and regular without gaps or breaks.}}

\newglossaryentry{correctness}{
	name={correctness, data},
	text={correctness},
	description={\index{Consistency!Self}self-\gls{consistency}, protection against alteration or corruption and \index{Consistency!With Requirements}\gls{consistency} with the functional requirements of the \gls{data-driven system}. IEC 61508 Part 3 \cite{citation:iec615083}}}

\newglossaryentry{coupling}{
	name={coupling (data)},
	text={coupling},
	description={The dependence of a software component on data not exclusively under the control of that software component. RTCA/DO-178C \cite{citation:ED12C}}} %doesn't seem to appear in text.

\newglossaryentry{criticality}{
	name={criticality, data},
	text={criticality},
	description={Classification of data by the potential effect of erroneous data on the expected operation that is supported by that data. RTCA/DO-200A \cite{citation:ED76}}}

\longnewglossaryentry{critical data}{
	name={critical data}}
	{\begin{itemize}
			\item Data whose loss of specific properties could contribute to a hazardous system state
			\item Data with an \index{Integrity Property}\gls{integrity} level as defined in Chapter 3, Section 3.2 point 3.2.8(a) of Annex 15 to the Chicago Convention, i.e., \index{Integrity Property}\gls{integrity} level one in one hundred million: there is a high probability when using corrupted critical data that the continued safe flight and landing of an aircraft would be severely at risk with the potential for catastrophe. (EU) No 73/2010 \cite{citation:EU732010}
	\end{itemize}}

\newglossaryentry{customisation data}{
	name={customisation data},
	description={Data used to configure a system or component. Def(Aust)5679 \cite{citation:DEFOz}}}

\longnewglossaryentry{data}{
	name={data},
	plural={data}}%
	{\begin{itemize}
		\item A thing given or granted; something known or assumed as fact, and made the basis of reasoning or calculation; an assumption or premiss from which inferences are drawn. \Gls{oed}
		\item A reinterpretable representation of \gls{information} in a formalized manner suitable for communication, interpretation or processing. ISO/IEC\ 2382 \cite{citation:ISO23821}
	\end{itemize}}
	
\newglossaryentry{database}{
	name={database},
	description={A set of data, part or the whole of another set of data, consisting of at least one file that is sufficient for a given purpose or for a given data processing system. RTCA/DO-178C}}

\longnewglossaryentry{data chain}{
	name={data chain}}%
	{\begin{itemize}
			\item An `Aeronautical Data Chain' is a conceptual representation of the path that a set, or element of aeronautical data takes from its creation to its end use. An aeronautical data chain is a series of interrelated links wherein each link provides a function that facilitates the origination, transmission and use of aeronautical data for a specific purpose. RTCA/DO-200A \cite{citation:ED76}
			\item A collection of organizational data processing functions, where data is transferred from one chain participant to another between data origination and end use. P. Ensor \cite{citation:Ensor2009}
			\item Any combination of two or more data elements, \glspl{item data}, data codes, and data abbreviations in a prescribed sequence to yield meaningful information; for example, ``date'' consists of data elements year, month, and day. McGraw-Hill Dictionary \cite{citation:McGrawHill}
	\end{itemize}}
	
\newglossaryentry{data dictionary}{
	name={data dictionary},
	plural={data dictionaries},
	description={The detailed description of data, parameters, variables, and constants used by the system. RTCA/DO-178C}}

\newglossaryentry{data-driven system}{
	name={data-driven system},
	description={System which relies upon \gls{configuration data} or lookup tables to define the functionality of the system. IEC 61508 Part 4 \cite{citation:iec615084}}}

\newglossaryentry{data-intensive system}{
	name={data-intensive system},
	description={Systems which make extensive use of large amounts of data. N. Storey \cite{citation:StoreyFaulkner20031}}}

\newglossaryentry{dataset}{name={dataset},description={Identifiable collection of data. Note that a dataset may be a smaller grouping of data which, though limited by some constraint such as spatial extent or feature type, is located physically within a larger dataset. Theoretically, a dataset may be as small as a single feature or feature attribute contained within a larger dataset. A hardcopy map or chart may be considered a dataset. BS EN ISO 19131:2008 \cite{citation:ISO19131}}}

\newglossaryentry{disposability}{name={disposability / deletability},description={The property that the data can be permanently removed when required}}

\longnewglossaryentry{error}{
	name={error, data},
	text={data error}}%
	{\begin{itemize}
			\item Discrepancy with the universe of discourse. ISO\ 19138:2006 \cite{citation:ISO19138}
			\item Discrepancy between a data value and the true, specified or theoretically correct value or condition. P. Ensor \cite{citation:Ensor2009}
	\end{itemize}}

\newglossaryentry{explainability}{name={explainability},description={The property that the data can be meaningfully explained, by a suitable mechanism, to those who need to understand it.}}

\newglossaryentry{fidelity}{
	name={fidelity / representation ,data},
	text={fidelity / representation}, 
	description={The property describing how well the data maps to the real world entity it is trying to model.}}

\newglossaryentry{format}{name={format}, description={The property that data is represented in a which is readable by those that need to use it.}}

\newglossaryentry{goldilocks}{name={Goldilocks},description={The property that the data is just the right size -- not to much and not too little}}

\newglossaryentry{hazard}{
	name={hazard, data},
	text={hazard},
	description={Use of data (in the context of a system) that could lead to harm. \gls{scsc} \gls{dsiwg}}}

\newglossaryentry{hazard log}{
	name={hazard log},
	description={A record of all hazard analysis, safety risk assessment and safety risk reduction activities for the ``whole-of-life'' of a safety-related system.}}

\newglossaryentry{history}{name={history}, description={Property that the data has an audit trail of changes.}}

\longnewglossaryentry{information}{
	name={information}}%
	{\begin{itemize}
			\item Knowledge communicated concerning some particular fact, subject, or event; that of which one is apprised or told - intelligence, news - as contrasted with data. \gls{oed}
			\item Knowledge that has a contextual meaning. ISO/IEC\ 2382  \cite{citation:ISO23821}
	\end{itemize}}

\newglossaryentry{information aeronautical}{
	name={information, aeronautical},
	text={aeronautical information},
	description={\Gls{information} resulting from the assembly, analysis and formatting of \gls{aeronautical data}. (EU) No 73/2010 \cite{citation:EU732010}}}

\longnewglossaryentry{integrity}{
	name={integrity, data},
	text={integrity}}%
	{\begin{itemize}
			\item The assurance that a data element retrieved from a storage system has not been corrupted or altered in any ways since the original data entry\index{Data!Entry} or latest authorised amendment. RTCA/DO-200A \cite{citation:ED76}
			\item The degree of assurance that a \gls{item data} and its value have not been lost or altered since the data origination or authorised amendment. (EU) No 73/2010 \cite{citation:EU732010}
			\item The degree of undetected (at system level) non-conformity of the input value of the \gls{item data} with its output value. (EU) No 1207/2011 \cite{citation:EU12072011}
			\item The property of protecting the \index{Accuracy Property}\gls{accuracy} and \gls{completeness} of assets, i.e., that which has value to the organization. ISO 27001:2013 \cite{citation:ISO27001:2013}
	\end{itemize}}

\newglossaryentry{intended_destination}{name={intended destination / usage},plural={intended desctinations / usages},description={Property that the data is only sent to those that should have access to it}}

\newglossaryentry{item data}{
	name={item, data},
	text={data item}, 
	description={Single attribute of a complete \gls{dataset}, which is allocated a value that defines its current status. (EU) No 73/2010 \cite{citation:EU732010}}}

\newglossaryentry{lifetime}{name={lifetime},description={The property of when the safety-related data expire}}

\newglossaryentry{metadata}{
	name={metadata},
	description={Data that represents \gls{information} about data itself. Note that one should distinguish between ``Structural Metadata'', which is data about the design and specification of data structures (and is more properly called ``data about the containers of data'') and ``Descriptive Metadata'', which is about individual instances of application data, the data content. J. Inge \cite{citation:inge2008improving}}}

\newglossaryentry{mitigation}{name={mitigation},description={Steps taken to control or prevent a hazard from causing harm and reduce risk to a tolerable or acceptable level. \cite{citation:esarr3}}}
	
\newglossaryentry{origination}{
	name={origination, data},
	text={origination},
	description={Creation of a new \gls{item data} with its associated value, the modification of the value of an existing \gls{item data} or the deletion of an existing \gls{item data}. (EU) No 73/2010 \cite{citation:EU732010}}}

\newglossaryentry{priority}{name={priority},plural={priorities},description={The property that data is presented / transmitted / made available in the order required.}}

\longnewglossaryentry{quality}{
	name={quality, data},
	text={data quality}}%
	{\begin{itemize}
			\item A degree or level of confidence that the data provided meet the requirements of the user. These requirements include levels of \gls{accuracy}\index{Accuracy Property}, \gls{resolution}\index{Resolution Property}, \gls{dsal}\index{Assurance Level}, \gls{traceability}, timeliness\index{Timeliness Property}, \gls{completeness}, and format. RTCA/DO-200A \cite{citation:ED76}
			\item A degree or level of confidence that the data provided meets the requirements of the data user in terms of \index{Accuracy Property}\gls{accuracy}, \gls{resolution}\index{Resolution Property} and \index{Integrity Property}\gls{integrity}. (EU) No 73/2010 \cite{citation:EU732010}
	\end{itemize}}

\longnewglossaryentry{resolution}{
	name={resolution}}%
	{\begin{itemize}
			\item The ability of a device to respond to small differences in input and to indicate or represent them accurately in output; a measure of this, expressed as the smallest difference so distinguishable. \gls{oed}.
			\item The smallest difference between two adjacent values that can be represented in a data storage, display or transfer system. RTCA/DO-200A \cite{citation:ED76}
			\item The number of units or digits to which a measured or calculated value is expressed and used. (EU) No 73/2010 \cite{citation:EU732010}
	\end{itemize}}

\newglossaryentry{routine data}{
	name={routine data},
	description={data with an \index{Integrity Property}\gls{integrity} level as defined in Chapter 3, Section 3.2 point 3.2.8(b) of Annex 15 to the Chicago Convention, i.e., \index{Integrity Property}\gls{integrity} level one in one thousand: there is a very low probability when using corrupted routine data that the continued safe flight and landing of an aircraft would be severely at risk with the potential for catastrophe. (EU) No 73/2010 \cite{citation:EU732010}}}

\newglossaryentry{sequencing}{
	name={sequencing},
	description={The property that the data is preserved in the order required.}}

\newglossaryentry{software lifecycle data}{
	name={software lifecycle data},
	description={Data that is produced during the software lifecycle\index{Lifecycle!Software} to plan, direct, explain, define, record, or provide evidence of activities (including the software product itself). This data enables the software lifecycle\index{Lifecycle!Software} processes, system or equipment approval and post-approval modification of the software product. ED-153 \cite{citation:ED153}}}

\newglossaryentry{suppression}{name={suppression},description={Property that the data is intended never to be used again}}

\newglossaryentry{tailoring}{name={tailoring},description={Adaptation of processes etc. to be appropriate for a specific system and context}}

\longnewglossaryentry{timeliness}{
	name={timeliness}}%
	{\begin{itemize}
			\item A measure of the time delay between a change in the real world and the associated \gls{database} update being available to the user. P. Ensor \cite{citation:Ensor2009}
			\item The difference between the time of output of a \gls{item data} and the time of applicability of that \gls{item data}. (EU) No 1207/2011 \cite{citation:EU12072011}
	\end{itemize}}

\newglossaryentry{traceability}{
	name={traceability},
	description={Ability to determine the origin of the data. RTCA/DO-200A \cite{citation:ED76}}}

\newglossaryentry{trace data}{
	name={trace data},
	description={Data providing evidence of \gls{traceability} of development and verification processes \index{Lifecycle!Software}\gls{software lifecycle data} data without implying the production of any particular artefact. Trace data may show linkages, for example, through the use of naming conventions or through the use of references or pointers either embedded in or external to the \index{Lifecycle!Software}\gls{software lifecycle data}. RTCA/DO-178C \cite{citation:ED12C}}}

\newglossaryentry{treat}{name={treat},description={To apply a \gls{treatment}}}

\longnewglossaryentry{validation}{
	name={validation, data},
	text={validation}}%
	{\begin{itemize}
			\item The activity whereby a data element is checked as having a value that is fully applicable to the identity given to the data element, or a set of data elements that is checked as being acceptable for their purpose. RTCA/DO-200A \cite{citation:ED76}
			\item Process of ensuring that data meets the requirements for the specified application or intended use. (EU) No 73/2010 \cite{citation:EU732010}
	\end{itemize}}

\newglossaryentry{validity}{
	name={validity, period of},
	text={validity},
	description={Period between the date and time on which information becomes available and the date and time on which the \gls{information} ceases to be effective. Based on (EU) No 73/2010 \cite{citation:EU732010}}}

\newglossaryentry{verification}{
	name={verification, data},
	text={verification},
	description={Evaluation of the output of a process to ensure \gls{correctness} and \index{Consistency!With Inputs}\gls{consistency} with respect to the inputs and applicable data standards, rules and conventions used in that process. Based on (EU) No 73/2010 \cite{citation:EU732010}}}

\newglossaryentry{verifiability}{
	name={verifiability},
	description={Evaluation of the output of an \gls{aeronautical data} process to ensure \gls{correctness} and \index{Consistency!With Inputs}\gls{consistency} with respect to the inputs and applicable data standards, rules and conventions used in that process. (EU) No 73/2010 \cite{citation:EU732010}}}