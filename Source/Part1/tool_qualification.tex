%================================================================================
%       Safety Critical Systems Club - Data Safety Initiative Working Group
%================================================================================
%                       DDDD    SSSS  IIIII  W   W   GGGG
%                       D   D  S        I    W   W  G   
%                       D   D   SSS     I    W W W  G  GG
%                       D   D      S    I    WW WW  G   G
%                       DDDD   SSSS   IIIII  W   W   GGG
%================================================================================
%               Data Safety Guidance Document - LaTeX Source File
%================================================================================
%
% Description:
%   Appendix regarding tool issues and selection
%================================================================================
\chapter{Tool Confidence and Tool Qualification of Data Processing Tools (Informative)}\index{Tools}\label{bkm:tools}
%
\dsiwgSectionQuote{Don't fix the blame, fix the problem}{Keith Pennington}

\section{Introduction}
A tool is a software that is required to build the product, but itself is not part of the product. For example, a compiler or a test tool.

\section{What the Standards say}
Main safety standards, like IEC\ 61508, ISO\ 26262, DO-178, EN-50128 require an assessment of ``tool confidence''
or the application of ``tool qualification'' to ensure that a lack of \gls{integrity} in the tools does not impact  the safety of the product. 

These safety standards use an approach which may be broadly defined as a three step process:

\begin{enumerate}
     \item Analyse the risks of the tool, so called ``tool classification''
     \item Make the tools safe, based on the tool classification, so called ``tool qualification'', usually by testing the tool.
       Note if a tool is classified as non-critical, for example, Tool Confidence Level 1 in ISO\ 26262, it does not need to be qualified.
     \item Use the tools as classified and qualified, such as by following a so called ``tool safety manual''
\end{enumerate}
 
The approach to classification differs in different standards and also the proposed qualification methods differ slightly,
but all share the same principle: use the tool carefully (safety manual) or qualify that the tool is doing what it should do.
More details can be found in the safety standards.

\section{Data tools}
 
For safety of data the same should be applied to all used tools -- they should be classified, eventually qualified and used according to the safety manuals.
 
A specific approach is not proposed here for data tools,
but a recommendation to use the standards with which a product has to be compliant.
For example, if data is to be used in the Automotive sector,
then consideration should be given to chapters 8--11 ``Confidence in the use of  software tools'' of ISO\ 26262.
 
The following generic recommendations are intended to manage tool confidence according to the standards applicable to the domain:

\begin{itemize}
\item The tool scope is wider: not only do tools for building the product have to be considered,
  but also the tools that create the data which impact the system,
  such as the maps of a self-driving car, or the training\index{Training!AI} data which is used to train \gls{ai} systems.
\item Consider all tools that are working with the data:
  \begin{itemize}
  \item Creation tools that create the data, for example, tools processing sensor data
  \item Data storing tools, that store the data, perhaps into a \gls{database}
  \item Data manipulation tools, that transform data, such as by translation to a different geodetic system, merging of different data sources or the creation of summaries
  \item Data visualization tools that show the data -- viewers that display the results
  \end{itemize}
  
\item Consider all outputs of the tools.
  For example an \gls{ai} training\index{Training!AI} tool usually creates two outputs:
  \begin{enumerate}
  \item The trained network and
  \item The achieved \gls{accuracy}.
  \end{enumerate}
  Each can be impacted by tool errors and therefore both features of the tool have to be classified/qualified  and used safely.
\item When working with generic tools like ``Excel'', ``SQL'', or ``Python'',
consider every new application as a new use of the tool -- in other words, classification of a tool can never be truly generic, but must be reviewed for each specific use. 
  An example of failing to carry out this reassessment is the Covid Summary Table (see \autoref{bkm:incacc:covidexcel}),
  which was written / created in Excel using some tables/schemes into which the data was imported and analysed.
\item The classification and qualification do not depend on the DSAL, however the DSAL should be considered when testing the tool or when classifying the risks.
  For example if a tool is to be qualified by testing, it should be done with the same rigour that is required  for the DSAL. 
\item When looking to the risks of tools, it is not sufficient to consider only the functional risks like ``wrong data created''
  (such as the potential that 1+1 could be computed to be 0 instead of 2).
  In addition, the data-specific risks specified in this document
  (\hyperref[tab:issues]{Section}~\ref{tab:issues} Ways that Data Can Cause  Problems) should be considered for impact from the tools.
  \Glspl{data property} that can be impacted from tools should also be considered.
  Therefore the following aspects of tools to data should be analysed during tool classification:
  \begin{itemize}
  \item Each issue listed in \dsiwgRef{subsection}{tab:issues}
  \item Each Property listed in \autoref{tab:PropertiesOfData}
  \end{itemize}    
\end{itemize}

For all above aspects the impact of the tool has to be considered, for example ``Interpretation'',
the risk of ``Wrong Interpretation'' can be expressed in the question:
``Can the tool change the data, such that it will be wrongly interpreted?''.
If this is not possible / makes no sense, then the error ``Wrong Interpretation'', can be classified as ``no impact''.
Otherwise the risk is valid and has either to be mitigated by a safety guideline in the safety manual,
or it should be excluded by systematic qualification  tests which show that the error ``Wrong Interpretation'' cannot occur.  
