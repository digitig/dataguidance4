%================================================================================
%       Safety Critical Systems Club - Data Safety Initiative Working Group
%================================================================================
%                       DDDD    SSSS  IIIII  W   W   GGGG
%                       D   D  S        I    W   W  G   
%                       D   D   SSS     I    W W W  G  GG
%                       D   D      S    I    WW WW  G   G
%                       DDDD   SSSS   IIIII  W   W   GGG
%================================================================================
%               Data Safety Guidance Document - LaTeX Source File
%================================================================================
%
% Description:
%   DSIWG History section.
%
%================================================================================
\section{\glsfmtshort{dsiwg} History (Discursive)} \label{bkm:history}

\dsiwgSectionQuote{If we have data, let's look at data. If all we have are opinions, let's go with mine.}{Jim Barksdale}

The task of developing generally applicable, pan-sector guidance for data safety issues was taken on by the (\gls{dsiwg}) of the \gls{scsc}. The \gls{dsiwg}'s work started with a seminar \textit{``How to Stop Data Causing Harm''}, which was held in December 2012; material from this seminar is available at: (\href{http://scsc.org.uk/e209}{http://scsc.org.uk/e209}) (accessed 30 October 2017). The first meeting of the group agreed the following vision: 

\begin{quote}
  \dsiwgTextBF{\dsiwgTextIT{To have clear guidance on how data (as distinct from software and hardware) should be managed in a safety-related context, which will reflect emerging best practice.}}
\end{quote}

The group, comprising industry, academics, government and independent consultants,
produced an initial guidance document in January 2014.
A subsequent version of the document was released in January 2015,
with a second seminar \dsiwgTextIT{''How to Stop Data Causing Harm: What You Need to Know''} held in December 2015;
material from this seminar is available at: \href{http://scsc.org.uk/e343}{http://scsc.org.uk/e343} (accessed 17 January 2023).
Details pertaining to a third data safety seminar \dsiwgTextIT{``Data Safety Evolution''} that was held in November 2019 may be found at
\href{https://scsc.uk/e651}{https://scsc.uk/e651}
(accessed 17 January 2023).
To help disseminate information, members of the \gls{dsiwg} have presented papers relating to data safety in a number of fora,
including the \gls{sss}.

Further revisions of the guidance document, which include new material generated during and as a consequence of \gls{dsiwg} meetings, were issued in January 2016, January 2017 and in January 2018.

The January 2018 release was version 3.0 and the \gls{dsiwg} decided to keep this version ``current'' for several years, to enable users to demonstrate compliance against a fixed target. However it was also necessary to keep the document up to date and address user feedback, particularly where this could improve the usability of the document. The next two versions were therefore kept in alignment with version 3.0, with all section numbers in the body of the document remaining constant.

The \docmodmonthlong \docmodyear is a complete rewrite of the guidance. The substance of the guidance remains unchanged, but more of the guidance is now normative so compliance with the guidance will need to be reviewed.
