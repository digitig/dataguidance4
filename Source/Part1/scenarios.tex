%================================================================================
%       Safety Critical Systems Club - Data Safety Initiative Working Group
%================================================================================
%                       DDDD    SSSS  IIIII  W   W   GGGG
%                       D   D  S        I    W   W  G   
%                       D   D   SSS     I    W W W  G  GG
%                       D   D      S    I    WW WW  G   G
%                       DDDD   SSSS   IIIII  W   W   GGG
%================================================================================
%               Data Safety Guidance Document - LaTeX Source File
%================================================================================
%
% Description:
%   Miscellanea section.
%
%================================================================================
\chapter{Scenarios\index{Scenario|textbf} (Discursive)} \label{bkm:scenario}

\dsiwgSectionQuote{Scenarios help solve problems differently because they can illuminate new possibilities and ignite hope.}{Roger Spitz}

Table \ref{tab:Scenarios} section identifies several scenarios under which it might be useful to recognise data as safety significant.

The levels of necessary assurance given in \todo{Insert reference}
depend on the relevant scenario, and these scenarios
are also used in subsequent chapters to indicate the relevance of
different tools and techniques. 
\begin{longtable}{|C{\dsiwgColumnWidth{0.06}}|L{\dsiwgColumnWidth{0.17}}|L{\dsiwgColumnWidth{0.77}}|}
	\caption{Data Safety Scenarios}
	\label{tab:Scenarios}
	\\\hline\TableHeadColourC{No.} & \TableHeadColour{Scenario} & \TableHeadColour{Description} \\\hline
	\endfirsthead
	\caption[]{Data Safety Scenarios (continued)}
	\\\hline\TableHeadColourC{No.} & \TableHeadColour{Scenario} & \TableHeadColour{Description} \\\hline
	\endhead
	\multicolumn{3}{r}{\sl Continued on next page}
	\endfoot\endlastfoot
	\hline
	S-0 & Check & A system is to be checked for new or changed data outputs or new or changed data dependencies.
	
	This could be a scheduled (regular) check or ad-hoc.\\
	\hline
	S-1 & New build & A system that uses or produces data is to be built largely from scratch.\\
	\hline
	S-2 & Assembly & A system that uses or produces data is to be built primarily using pre-existing components.\\
	\hline
	S-3 & Restructure & An existing system that uses or produces data is to undergo major restructuring. \\
	\hline
	S-4 & Assure & An assurance position is to be created for an existing system that uses or produces data to mitigate potential safety concerns.
	
	This may be due to impending deployment, change of use, change to the way problems are handled, changing legislation or increased public concern. This also occurs when a system has been incrementally extended (or combined with other systems).\\
	\hline
	S-5 & New use & An existing system which uses or produces data, which was not safety-affecting, is to be deployed in a context where it will be safety-affecting.\\
	\hline
	S-6 & Degraded & As a result of a failure, a system which uses or produces data is running in degraded mode and this level of operation is likely to persist for some time.
	
	A new safety analysis and position need to be created.\\
	\hline
	S-7 & Investigation & A significant safety-affecting incident has occurred.
	
	The cause of this needs to be understood and mitigations proposed to reduce the risk of such an incident in the future.\\
	\hline
	S-8 & Migrate & Data is to be migrated from one system to another. This may involve data selection, cleansing and transformation. \\
	\hline
	S-9 & Combine & Data sets are combined within a system to produce new resulting data.\\
	\hline
	S-10 & Cleanse & \\
	\hline
	S-11 & Audit & \\
	\hline
	S-12 & Filter & Data is to be filtered or selected within a system to produce a new subset ot data\\
	\hline
\end{longtable}\todo{Provide description of ``Cleanse'' and ``Audit''}

